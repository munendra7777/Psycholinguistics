
% Default to the notebook output style

    


% Inherit from the specified cell style.




    
\documentclass[11pt]{article}

    
    
    \usepackage[T1]{fontenc}
    % Nicer default font (+ math font) than Computer Modern for most use cases
    \usepackage{mathpazo}

    % Basic figure setup, for now with no caption control since it's done
    % automatically by Pandoc (which extracts ![](path) syntax from Markdown).
    \usepackage{graphicx}
    % We will generate all images so they have a width \maxwidth. This means
    % that they will get their normal width if they fit onto the page, but
    % are scaled down if they would overflow the margins.
    \makeatletter
    \def\maxwidth{\ifdim\Gin@nat@width>\linewidth\linewidth
    \else\Gin@nat@width\fi}
    \makeatother
    \let\Oldincludegraphics\includegraphics
    % Set max figure width to be 80% of text width, for now hardcoded.
    \renewcommand{\includegraphics}[1]{\Oldincludegraphics[width=.8\maxwidth]{#1}}
    % Ensure that by default, figures have no caption (until we provide a
    % proper Figure object with a Caption API and a way to capture that
    % in the conversion process - todo).
    \usepackage{caption}
    \DeclareCaptionLabelFormat{nolabel}{}
    \captionsetup{labelformat=nolabel}

    \usepackage{adjustbox} % Used to constrain images to a maximum size 
    \usepackage{xcolor} % Allow colors to be defined
    \usepackage{enumerate} % Needed for markdown enumerations to work
    \usepackage{geometry} % Used to adjust the document margins
    \usepackage{amsmath} % Equations
    \usepackage{amssymb} % Equations
    \usepackage{textcomp} % defines textquotesingle
    % Hack from http://tex.stackexchange.com/a/47451/13684:
    \AtBeginDocument{%
        \def\PYZsq{\textquotesingle}% Upright quotes in Pygmentized code
    }
    \usepackage{upquote} % Upright quotes for verbatim code
    \usepackage{eurosym} % defines \euro
    \usepackage[mathletters]{ucs} % Extended unicode (utf-8) support
    \usepackage[utf8x]{inputenc} % Allow utf-8 characters in the tex document
    \usepackage{fancyvrb} % verbatim replacement that allows latex
    \usepackage{grffile} % extends the file name processing of package graphics 
                         % to support a larger range 
    % The hyperref package gives us a pdf with properly built
    % internal navigation ('pdf bookmarks' for the table of contents,
    % internal cross-reference links, web links for URLs, etc.)
    \usepackage{hyperref}
    \usepackage{longtable} % longtable support required by pandoc >1.10
    \usepackage{booktabs}  % table support for pandoc > 1.12.2
    \usepackage[inline]{enumitem} % IRkernel/repr support (it uses the enumerate* environment)
    \usepackage[normalem]{ulem} % ulem is needed to support strikethroughs (\sout)
                                % normalem makes italics be italics, not underlines
    \usepackage{mathrsfs}
    

    
    
    % Colors for the hyperref package
    \definecolor{urlcolor}{rgb}{0,.145,.698}
    \definecolor{linkcolor}{rgb}{.71,0.21,0.01}
    \definecolor{citecolor}{rgb}{.12,.54,.11}

    % ANSI colors
    \definecolor{ansi-black}{HTML}{3E424D}
    \definecolor{ansi-black-intense}{HTML}{282C36}
    \definecolor{ansi-red}{HTML}{E75C58}
    \definecolor{ansi-red-intense}{HTML}{B22B31}
    \definecolor{ansi-green}{HTML}{00A250}
    \definecolor{ansi-green-intense}{HTML}{007427}
    \definecolor{ansi-yellow}{HTML}{DDB62B}
    \definecolor{ansi-yellow-intense}{HTML}{B27D12}
    \definecolor{ansi-blue}{HTML}{208FFB}
    \definecolor{ansi-blue-intense}{HTML}{0065CA}
    \definecolor{ansi-magenta}{HTML}{D160C4}
    \definecolor{ansi-magenta-intense}{HTML}{A03196}
    \definecolor{ansi-cyan}{HTML}{60C6C8}
    \definecolor{ansi-cyan-intense}{HTML}{258F8F}
    \definecolor{ansi-white}{HTML}{C5C1B4}
    \definecolor{ansi-white-intense}{HTML}{A1A6B2}
    \definecolor{ansi-default-inverse-fg}{HTML}{FFFFFF}
    \definecolor{ansi-default-inverse-bg}{HTML}{000000}

    % commands and environments needed by pandoc snippets
    % extracted from the output of `pandoc -s`
    \providecommand{\tightlist}{%
      \setlength{\itemsep}{0pt}\setlength{\parskip}{0pt}}
    \DefineVerbatimEnvironment{Highlighting}{Verbatim}{commandchars=\\\{\}}
    % Add ',fontsize=\small' for more characters per line
    \newenvironment{Shaded}{}{}
    \newcommand{\KeywordTok}[1]{\textcolor[rgb]{0.00,0.44,0.13}{\textbf{{#1}}}}
    \newcommand{\DataTypeTok}[1]{\textcolor[rgb]{0.56,0.13,0.00}{{#1}}}
    \newcommand{\DecValTok}[1]{\textcolor[rgb]{0.25,0.63,0.44}{{#1}}}
    \newcommand{\BaseNTok}[1]{\textcolor[rgb]{0.25,0.63,0.44}{{#1}}}
    \newcommand{\FloatTok}[1]{\textcolor[rgb]{0.25,0.63,0.44}{{#1}}}
    \newcommand{\CharTok}[1]{\textcolor[rgb]{0.25,0.44,0.63}{{#1}}}
    \newcommand{\StringTok}[1]{\textcolor[rgb]{0.25,0.44,0.63}{{#1}}}
    \newcommand{\CommentTok}[1]{\textcolor[rgb]{0.38,0.63,0.69}{\textit{{#1}}}}
    \newcommand{\OtherTok}[1]{\textcolor[rgb]{0.00,0.44,0.13}{{#1}}}
    \newcommand{\AlertTok}[1]{\textcolor[rgb]{1.00,0.00,0.00}{\textbf{{#1}}}}
    \newcommand{\FunctionTok}[1]{\textcolor[rgb]{0.02,0.16,0.49}{{#1}}}
    \newcommand{\RegionMarkerTok}[1]{{#1}}
    \newcommand{\ErrorTok}[1]{\textcolor[rgb]{1.00,0.00,0.00}{\textbf{{#1}}}}
    \newcommand{\NormalTok}[1]{{#1}}
    
    % Additional commands for more recent versions of Pandoc
    \newcommand{\ConstantTok}[1]{\textcolor[rgb]{0.53,0.00,0.00}{{#1}}}
    \newcommand{\SpecialCharTok}[1]{\textcolor[rgb]{0.25,0.44,0.63}{{#1}}}
    \newcommand{\VerbatimStringTok}[1]{\textcolor[rgb]{0.25,0.44,0.63}{{#1}}}
    \newcommand{\SpecialStringTok}[1]{\textcolor[rgb]{0.73,0.40,0.53}{{#1}}}
    \newcommand{\ImportTok}[1]{{#1}}
    \newcommand{\DocumentationTok}[1]{\textcolor[rgb]{0.73,0.13,0.13}{\textit{{#1}}}}
    \newcommand{\AnnotationTok}[1]{\textcolor[rgb]{0.38,0.63,0.69}{\textbf{\textit{{#1}}}}}
    \newcommand{\CommentVarTok}[1]{\textcolor[rgb]{0.38,0.63,0.69}{\textbf{\textit{{#1}}}}}
    \newcommand{\VariableTok}[1]{\textcolor[rgb]{0.10,0.09,0.49}{{#1}}}
    \newcommand{\ControlFlowTok}[1]{\textcolor[rgb]{0.00,0.44,0.13}{\textbf{{#1}}}}
    \newcommand{\OperatorTok}[1]{\textcolor[rgb]{0.40,0.40,0.40}{{#1}}}
    \newcommand{\BuiltInTok}[1]{{#1}}
    \newcommand{\ExtensionTok}[1]{{#1}}
    \newcommand{\PreprocessorTok}[1]{\textcolor[rgb]{0.74,0.48,0.00}{{#1}}}
    \newcommand{\AttributeTok}[1]{\textcolor[rgb]{0.49,0.56,0.16}{{#1}}}
    \newcommand{\InformationTok}[1]{\textcolor[rgb]{0.38,0.63,0.69}{\textbf{\textit{{#1}}}}}
    \newcommand{\WarningTok}[1]{\textcolor[rgb]{0.38,0.63,0.69}{\textbf{\textit{{#1}}}}}
    
    
    % Define a nice break command that doesn't care if a line doesn't already
    % exist.
    \def\br{\hspace*{\fill} \\* }
    % Math Jax compatibility definitions
    \def\gt{>}
    \def\lt{<}
    \let\Oldtex\TeX
    \let\Oldlatex\LaTeX
    \renewcommand{\TeX}{\textrm{\Oldtex}}
    \renewcommand{\LaTeX}{\textrm{\Oldlatex}}
    % Document parameters
    % Document title
    \title{Untitled1}
    
    
    
    
    

    % Pygments definitions
    
\makeatletter
\def\PY@reset{\let\PY@it=\relax \let\PY@bf=\relax%
    \let\PY@ul=\relax \let\PY@tc=\relax%
    \let\PY@bc=\relax \let\PY@ff=\relax}
\def\PY@tok#1{\csname PY@tok@#1\endcsname}
\def\PY@toks#1+{\ifx\relax#1\empty\else%
    \PY@tok{#1}\expandafter\PY@toks\fi}
\def\PY@do#1{\PY@bc{\PY@tc{\PY@ul{%
    \PY@it{\PY@bf{\PY@ff{#1}}}}}}}
\def\PY#1#2{\PY@reset\PY@toks#1+\relax+\PY@do{#2}}

\expandafter\def\csname PY@tok@gd\endcsname{\def\PY@tc##1{\textcolor[rgb]{0.63,0.00,0.00}{##1}}}
\expandafter\def\csname PY@tok@gu\endcsname{\let\PY@bf=\textbf\def\PY@tc##1{\textcolor[rgb]{0.50,0.00,0.50}{##1}}}
\expandafter\def\csname PY@tok@gt\endcsname{\def\PY@tc##1{\textcolor[rgb]{0.00,0.27,0.87}{##1}}}
\expandafter\def\csname PY@tok@gs\endcsname{\let\PY@bf=\textbf}
\expandafter\def\csname PY@tok@gr\endcsname{\def\PY@tc##1{\textcolor[rgb]{1.00,0.00,0.00}{##1}}}
\expandafter\def\csname PY@tok@cm\endcsname{\let\PY@it=\textit\def\PY@tc##1{\textcolor[rgb]{0.25,0.50,0.50}{##1}}}
\expandafter\def\csname PY@tok@vg\endcsname{\def\PY@tc##1{\textcolor[rgb]{0.10,0.09,0.49}{##1}}}
\expandafter\def\csname PY@tok@vi\endcsname{\def\PY@tc##1{\textcolor[rgb]{0.10,0.09,0.49}{##1}}}
\expandafter\def\csname PY@tok@vm\endcsname{\def\PY@tc##1{\textcolor[rgb]{0.10,0.09,0.49}{##1}}}
\expandafter\def\csname PY@tok@mh\endcsname{\def\PY@tc##1{\textcolor[rgb]{0.40,0.40,0.40}{##1}}}
\expandafter\def\csname PY@tok@cs\endcsname{\let\PY@it=\textit\def\PY@tc##1{\textcolor[rgb]{0.25,0.50,0.50}{##1}}}
\expandafter\def\csname PY@tok@ge\endcsname{\let\PY@it=\textit}
\expandafter\def\csname PY@tok@vc\endcsname{\def\PY@tc##1{\textcolor[rgb]{0.10,0.09,0.49}{##1}}}
\expandafter\def\csname PY@tok@il\endcsname{\def\PY@tc##1{\textcolor[rgb]{0.40,0.40,0.40}{##1}}}
\expandafter\def\csname PY@tok@go\endcsname{\def\PY@tc##1{\textcolor[rgb]{0.53,0.53,0.53}{##1}}}
\expandafter\def\csname PY@tok@cp\endcsname{\def\PY@tc##1{\textcolor[rgb]{0.74,0.48,0.00}{##1}}}
\expandafter\def\csname PY@tok@gi\endcsname{\def\PY@tc##1{\textcolor[rgb]{0.00,0.63,0.00}{##1}}}
\expandafter\def\csname PY@tok@gh\endcsname{\let\PY@bf=\textbf\def\PY@tc##1{\textcolor[rgb]{0.00,0.00,0.50}{##1}}}
\expandafter\def\csname PY@tok@ni\endcsname{\let\PY@bf=\textbf\def\PY@tc##1{\textcolor[rgb]{0.60,0.60,0.60}{##1}}}
\expandafter\def\csname PY@tok@nl\endcsname{\def\PY@tc##1{\textcolor[rgb]{0.63,0.63,0.00}{##1}}}
\expandafter\def\csname PY@tok@nn\endcsname{\let\PY@bf=\textbf\def\PY@tc##1{\textcolor[rgb]{0.00,0.00,1.00}{##1}}}
\expandafter\def\csname PY@tok@no\endcsname{\def\PY@tc##1{\textcolor[rgb]{0.53,0.00,0.00}{##1}}}
\expandafter\def\csname PY@tok@na\endcsname{\def\PY@tc##1{\textcolor[rgb]{0.49,0.56,0.16}{##1}}}
\expandafter\def\csname PY@tok@nb\endcsname{\def\PY@tc##1{\textcolor[rgb]{0.00,0.50,0.00}{##1}}}
\expandafter\def\csname PY@tok@nc\endcsname{\let\PY@bf=\textbf\def\PY@tc##1{\textcolor[rgb]{0.00,0.00,1.00}{##1}}}
\expandafter\def\csname PY@tok@nd\endcsname{\def\PY@tc##1{\textcolor[rgb]{0.67,0.13,1.00}{##1}}}
\expandafter\def\csname PY@tok@ne\endcsname{\let\PY@bf=\textbf\def\PY@tc##1{\textcolor[rgb]{0.82,0.25,0.23}{##1}}}
\expandafter\def\csname PY@tok@nf\endcsname{\def\PY@tc##1{\textcolor[rgb]{0.00,0.00,1.00}{##1}}}
\expandafter\def\csname PY@tok@si\endcsname{\let\PY@bf=\textbf\def\PY@tc##1{\textcolor[rgb]{0.73,0.40,0.53}{##1}}}
\expandafter\def\csname PY@tok@s2\endcsname{\def\PY@tc##1{\textcolor[rgb]{0.73,0.13,0.13}{##1}}}
\expandafter\def\csname PY@tok@nt\endcsname{\let\PY@bf=\textbf\def\PY@tc##1{\textcolor[rgb]{0.00,0.50,0.00}{##1}}}
\expandafter\def\csname PY@tok@nv\endcsname{\def\PY@tc##1{\textcolor[rgb]{0.10,0.09,0.49}{##1}}}
\expandafter\def\csname PY@tok@s1\endcsname{\def\PY@tc##1{\textcolor[rgb]{0.73,0.13,0.13}{##1}}}
\expandafter\def\csname PY@tok@dl\endcsname{\def\PY@tc##1{\textcolor[rgb]{0.73,0.13,0.13}{##1}}}
\expandafter\def\csname PY@tok@ch\endcsname{\let\PY@it=\textit\def\PY@tc##1{\textcolor[rgb]{0.25,0.50,0.50}{##1}}}
\expandafter\def\csname PY@tok@m\endcsname{\def\PY@tc##1{\textcolor[rgb]{0.40,0.40,0.40}{##1}}}
\expandafter\def\csname PY@tok@gp\endcsname{\let\PY@bf=\textbf\def\PY@tc##1{\textcolor[rgb]{0.00,0.00,0.50}{##1}}}
\expandafter\def\csname PY@tok@sh\endcsname{\def\PY@tc##1{\textcolor[rgb]{0.73,0.13,0.13}{##1}}}
\expandafter\def\csname PY@tok@ow\endcsname{\let\PY@bf=\textbf\def\PY@tc##1{\textcolor[rgb]{0.67,0.13,1.00}{##1}}}
\expandafter\def\csname PY@tok@sx\endcsname{\def\PY@tc##1{\textcolor[rgb]{0.00,0.50,0.00}{##1}}}
\expandafter\def\csname PY@tok@bp\endcsname{\def\PY@tc##1{\textcolor[rgb]{0.00,0.50,0.00}{##1}}}
\expandafter\def\csname PY@tok@c1\endcsname{\let\PY@it=\textit\def\PY@tc##1{\textcolor[rgb]{0.25,0.50,0.50}{##1}}}
\expandafter\def\csname PY@tok@fm\endcsname{\def\PY@tc##1{\textcolor[rgb]{0.00,0.00,1.00}{##1}}}
\expandafter\def\csname PY@tok@o\endcsname{\def\PY@tc##1{\textcolor[rgb]{0.40,0.40,0.40}{##1}}}
\expandafter\def\csname PY@tok@kc\endcsname{\let\PY@bf=\textbf\def\PY@tc##1{\textcolor[rgb]{0.00,0.50,0.00}{##1}}}
\expandafter\def\csname PY@tok@c\endcsname{\let\PY@it=\textit\def\PY@tc##1{\textcolor[rgb]{0.25,0.50,0.50}{##1}}}
\expandafter\def\csname PY@tok@mf\endcsname{\def\PY@tc##1{\textcolor[rgb]{0.40,0.40,0.40}{##1}}}
\expandafter\def\csname PY@tok@err\endcsname{\def\PY@bc##1{\setlength{\fboxsep}{0pt}\fcolorbox[rgb]{1.00,0.00,0.00}{1,1,1}{\strut ##1}}}
\expandafter\def\csname PY@tok@mb\endcsname{\def\PY@tc##1{\textcolor[rgb]{0.40,0.40,0.40}{##1}}}
\expandafter\def\csname PY@tok@ss\endcsname{\def\PY@tc##1{\textcolor[rgb]{0.10,0.09,0.49}{##1}}}
\expandafter\def\csname PY@tok@sr\endcsname{\def\PY@tc##1{\textcolor[rgb]{0.73,0.40,0.53}{##1}}}
\expandafter\def\csname PY@tok@mo\endcsname{\def\PY@tc##1{\textcolor[rgb]{0.40,0.40,0.40}{##1}}}
\expandafter\def\csname PY@tok@kd\endcsname{\let\PY@bf=\textbf\def\PY@tc##1{\textcolor[rgb]{0.00,0.50,0.00}{##1}}}
\expandafter\def\csname PY@tok@mi\endcsname{\def\PY@tc##1{\textcolor[rgb]{0.40,0.40,0.40}{##1}}}
\expandafter\def\csname PY@tok@kn\endcsname{\let\PY@bf=\textbf\def\PY@tc##1{\textcolor[rgb]{0.00,0.50,0.00}{##1}}}
\expandafter\def\csname PY@tok@cpf\endcsname{\let\PY@it=\textit\def\PY@tc##1{\textcolor[rgb]{0.25,0.50,0.50}{##1}}}
\expandafter\def\csname PY@tok@kr\endcsname{\let\PY@bf=\textbf\def\PY@tc##1{\textcolor[rgb]{0.00,0.50,0.00}{##1}}}
\expandafter\def\csname PY@tok@s\endcsname{\def\PY@tc##1{\textcolor[rgb]{0.73,0.13,0.13}{##1}}}
\expandafter\def\csname PY@tok@kp\endcsname{\def\PY@tc##1{\textcolor[rgb]{0.00,0.50,0.00}{##1}}}
\expandafter\def\csname PY@tok@w\endcsname{\def\PY@tc##1{\textcolor[rgb]{0.73,0.73,0.73}{##1}}}
\expandafter\def\csname PY@tok@kt\endcsname{\def\PY@tc##1{\textcolor[rgb]{0.69,0.00,0.25}{##1}}}
\expandafter\def\csname PY@tok@sc\endcsname{\def\PY@tc##1{\textcolor[rgb]{0.73,0.13,0.13}{##1}}}
\expandafter\def\csname PY@tok@sb\endcsname{\def\PY@tc##1{\textcolor[rgb]{0.73,0.13,0.13}{##1}}}
\expandafter\def\csname PY@tok@sa\endcsname{\def\PY@tc##1{\textcolor[rgb]{0.73,0.13,0.13}{##1}}}
\expandafter\def\csname PY@tok@k\endcsname{\let\PY@bf=\textbf\def\PY@tc##1{\textcolor[rgb]{0.00,0.50,0.00}{##1}}}
\expandafter\def\csname PY@tok@se\endcsname{\let\PY@bf=\textbf\def\PY@tc##1{\textcolor[rgb]{0.73,0.40,0.13}{##1}}}
\expandafter\def\csname PY@tok@sd\endcsname{\let\PY@it=\textit\def\PY@tc##1{\textcolor[rgb]{0.73,0.13,0.13}{##1}}}

\def\PYZbs{\char`\\}
\def\PYZus{\char`\_}
\def\PYZob{\char`\{}
\def\PYZcb{\char`\}}
\def\PYZca{\char`\^}
\def\PYZam{\char`\&}
\def\PYZlt{\char`\<}
\def\PYZgt{\char`\>}
\def\PYZsh{\char`\#}
\def\PYZpc{\char`\%}
\def\PYZdl{\char`\$}
\def\PYZhy{\char`\-}
\def\PYZsq{\char`\'}
\def\PYZdq{\char`\"}
\def\PYZti{\char`\~}
% for compatibility with earlier versions
\def\PYZat{@}
\def\PYZlb{[}
\def\PYZrb{]}
\makeatother


    % Exact colors from NB
    \definecolor{incolor}{rgb}{0.0, 0.0, 0.5}
    \definecolor{outcolor}{rgb}{0.545, 0.0, 0.0}



    
    % Prevent overflowing lines due to hard-to-break entities
    \sloppy 
    % Setup hyperref package
    \hypersetup{
      breaklinks=true,  % so long urls are correctly broken across lines
      colorlinks=true,
      urlcolor=urlcolor,
      linkcolor=linkcolor,
      citecolor=citecolor,
      }
    % Slightly bigger margins than the latex defaults
    
    \geometry{verbose,tmargin=1in,bmargin=1in,lmargin=1in,rmargin=1in}
    
    

    \begin{document}
    
    
    \maketitle
    
    

    
    \begin{itemize}
\tightlist
\item
  {[}x{]} \textbf{Munendra Kumar}
\item
  {[}X{]} \textbf{HOOMACL20170010}
\item
  {[}X{]} \textbf{M.A.~Computational Linguistics}
\item
  {[}X{]} \textbf{Semester 4}
\item
  {[}X{]} \textbf{LS363: Seminar in Psycholinguistics}
\item
  {[}X{]} \textbf{Course Instructor: Dr.~Shruti Sircar}
\end{itemize}

    \hypertarget{seminar-in-psycholinguistics}{%
\section{Seminar in
Psycholinguistics}\label{seminar-in-psycholinguistics}}

\hypertarget{internal-assesment-i}{%
\subsection{Internal Assesment I}\label{internal-assesment-i}}

    \hypertarget{section-1}{%
\subsubsection{Section 1}\label{section-1}}

\hypertarget{what-are-some-of-the-major-theories-in-cognitive-development-how-does-spelkes-core-knowledge-theory-compare-with-piagets-theory}{%
\paragraph{What are some of the major theories in cognitive development?
How does Spelke's core knowledge theory compare with Piaget's
theory?}\label{what-are-some-of-the-major-theories-in-cognitive-development-how-does-spelkes-core-knowledge-theory-compare-with-piagets-theory}}

To understand \textbf{cognitive development} we must fist understand
what \textbf{Cognition} Is. Cognition refers to all activity, processes,
and products of the mind ,for example, Whatever you perceive using your
senses. \emph{Piaget's (1936)} \textbf{theory of cognitive development}
explains how a child constructs a mental picture of the world. He is
also known as \textbf{father of psychology}. He was a contructivist
i.e.~his theory suggest that children learn thing in building blocks as
in constructing knowledge in bits for themselves. He observed his own
children in infancy. Certainly, child perceives the world as it is and
is very different from how how adults see it. He disagreed with the idea
that intelligence was a fixed trait, and regarded cognitive development
as a process which occurs due to biological maturation and interaction
with the outer world.

\emph{Piaget }was the first psychologist to make a systematic study of
cognitive development and before him no recognisable work of cognitive
development existed and the common assumption in psychology was that
children are merely less competent thinkers than adults. Piaget showed
that young children think in amazingly different ways compared to
adults.Even though the theories proposed by \emph{Piaget} dates back to
the decades but his earliest research is still informative. The
longitivity of his theories have many reasons. Some of them are listed
below:

\begin{itemize}
\tightlist
\item
  His theories/concepts provides us the sense as to what children's
  thinking is like.
\item
  His theory addressses the topics that have been interest of many
  scholars, scientists, philosophers and parents
\item
  His theory covers a wide age span of child development i.e.~from
  infancy to adolescence
\item
  His theory covers object permanance which was the resultant of his
  observation concerning infants' failure to search for objects if they
  cannot see it.
\end{itemize}

    \emph{Immanual Kant} was his major motivation for the development of his
theories, who was also interested in \emph{Origin of Knowledge}. His
contributions include a stage theory of child cognitive development,
detailed observational studies of cognition in children, and a series of
simple but ingenious tests to reveal different cognitive abilities.

\begin{itemize}
\item
  \textbf{There are three basic components to Piaget's cognitive
  theory:}
\item
  \textbf{Schemas} Schemas are the basic building blocks of such
  cognitive models, and enable us to form a mental representation of the
  world. Piaget (1952, p.~7) defined a schema as: \textgreater{} ``a
  cohesive, repeatable action sequence possessing component actions that
  are tightly interconnected and governed by a core meaning.''
\item
  \textbf{Adaptation processes that enable the transition from one stage
  to another}
\item
  \textbf{assimilation} (i.e.~using an existing schema (knowledge) to
  deal with a new object or situation) - \textbf{equilibrium} (occurs
  when a child's schemas can deal with most new information through
  assimilation)

  \begin{itemize}
  \tightlist
  \item
    \textbf{accomodation} (occuIntroductionrs when the existing schema
    does not work, and needs to be changed to deal with a new object or
    situation)
  \end{itemize}
\end{itemize}

\begin{figure}
\centering
\includegraphics{attachment:piaget-adaptation.jpg}
\caption{piaget-adaptation.jpg}
\end{figure}

\emph{For xample: a toddler sees a }kite* in the sky and calls it a bird
from its previous schema and calls it a bird and mother corrects him and
tells its not a bird but its a \emph{kite} and its a non-living object.
This state is known as \emph{assimilation}. The toddler now understands
the difference between \emph{the bird} and \emph{the kite}. This state
is known as state of \emph{disequilibrium} and the next time the toddler
sees a \emph{kite} and recognises it as a \emph{kite} This state is
known as \emph{accomodation}. {[}Note:Assimilation is the addition of
entirely new concept/idea/object while accomodation is the adjustment to
the previuosly assimilated concept.{]}

    \begin{itemize}
\tightlist
\item
  \textbf{Stages of Development}
\end{itemize}

According to \emph{Piaget} all children walkthrough four stages and they
do so in the same order. The four stages are as follows:

\begin{itemize}
\tightlist
\item
  \textbf{Sensorimotor}: (ages 0-2), thinking involves seeing, hearing,
  moving, touching, tasting, etc.
\end{itemize}

During this stage, children go from reflex-type actions to goal-directed
activity and physically interact with objects to move to next stage. At
this stage \emph{for example, children learn how to adjust their pupil
for the light sensitivity.}

\begin{itemize}
\item
  \textbf{Preoperational}: (ages 2-7), Development of language and the
  ability to use symbols.

  Child is self-centric; has trouble understanding another person's
  point of view. At this stage they learn to represent things
  symbolically i.e.~mental imagery,drawing or language acquisition.
  \emph{for example, at this stage children usually develop
  egocentricism in them i.e.~they cannot think of another aspect or
  cannot accomodate someone else's point of view.}
\item
  \textbf{Concrete Operational}: (ages 7-11), Hands-on thinking stage.
  Child understands reversibility, the laws of conservation, and can
  classify and seriate things. \emph{for example, at this stage children
  are able to put themselves in someone else's shoes i.e.~they can take
  other's point of view or take into account more than one perspective.}

  \begin{itemize}
  \tightlist
  \item
    \textbf{Formal Operational}: (ages 11-adult), develops the ability
    to think abstractly, think outside themselves about issues and
    problems in society, and works to develop an identity. This stage
    brings up the child's potential for solving different types of
    problems that were impossible for them tackle at earlier stages. He
    says that this stage is not universal. \emph{for example, children
    at this stage understand that there are more realities might lead
    their interest to fiction books.}
  \end{itemize}
\end{itemize}

He came to the realization that, children view the world differently
than adults do. And, there seemed to be stages you could look at where
the way they view the world changes in some systematic way. In fact he
had this notion that, there's a certain ability with each of these
stages that they have to grasp, and until they grasp that, they can't
reach the next level and in order to properly educate them we need to
detrmine at what stage they actually are. We cannot not just treat them
like little adults, it won't work, because they don't think like little
adults. So, he spent a lot of his life describing these stages. He
theorized that the development of thinking and language can be traced
back to actions, perceptions, and imitations by babies.

    \textbf{Sociocultural Approach} Russian psychologist \emph{Lev Vygotsky}
portrayed children as social beings intertwined with other people who
were eager to help them learn and gain skills. According to
\emph{Vygotsky} there is a strong connection between learning language
and the development of thinking. He states that higher mental processes
are first developed through interaction and shared activities between
children and another person usually parent/sibling. Children can then
internalize these processes.

\emph{Vygotsky}, unlike \emph{Piaget}, thought that abstract thinking
could not develop on its own, but required language and Western
schooling. Vygotsky believed that development and learning worked
together through socialization and language and also that there was
conPost \emph{Piaget} there is no grand unifying theory of cognitive
development. However there have been several trends post piaget in
developmental reasearch:

\begin{itemize}
\tightlist
\item
  connectionism (Bates, Elman, McClelland, Munakata)
\item
  information processing approaches (Case, Klahr, Siegler)
\item
  modularity theories (Leslie)
\item
  core knowledge (Spelke, Carey)
\item
  theory theory (Spelke, Carey, Gelman, Gopnik, Meltzoff,Wellman)
\end{itemize}

We will look at \emph{core knowledge(Spelke, Carey)} as that satisfies
our purpose of question. Core knowledge theory addresses that human
reasoning is guided by a collection of innate domain specific systems of
knowledge. According to \emph{Spelke} children are like \emph{primate
scientists} and argues that children are much more advanced in their
thinking than \emph{Piaget} estimated. These core knowledge systems are
innate mechanisms that do specific tasks. *For example: a songbird's
ability to learn a song characteristic of their species or an ant's
ability to navigate a terrain in search of food, and a human's ability
to acquire language for that matter.

Core knowledge theory operates on two domains:

\begin{itemize}
\item
  \textbf{The domain of objects}
\item
  Experiments with infants show that from an early age infants parse the
  visual array into objects and events. As early as 2 months of age,
  infants expect objects to be permanent *For Example: If a ball is kept
  out of sight of an infant while playing he will look for the ball
  around him.
\item
  and that two objects cannot occupy the same space at the same time.
  \emph{For example: a ball cannot pass through a wall.}
\end{itemize}

Objects are perceived as bounded, solid, continuous entities.

\begin{itemize}
\tightlist
\item
  \textbf{The domain of number}
\item
  Similar to our knowledge of objects we also posses knowlege of
  knowledge. Experiments with infants show that from as early as 6
  months of age, infants can detect a 1:2 ratio change in large numbers,
  but not a 2:3 ratio changen i.e.~they would be able to tell the
  diffrence between 10 balls and 20 marbles but not between the 20 and
  30 marbles. By 9 months of age, infants can detect a 2:3 ratio change
  and the precision in detection increases to a 7:8 ratio in adults.
\end{itemize}

    The core knowledge domains have mainly three characteristics: - core
knowledge is task specific such that each system functions to solve a
limited set of problems. \emph{For example, the number system can
discriminate quantities based on their ratio, but this is not equivalent
to the ability to count. The natural number system that supports
counting is a complex construction that requires integration from the
object and number systems as well as knowledge about the relation
between number words and their referents.}

\begin{itemize}
\item
  core knowledge domains are summarised in such a way that each system
  operates with a fair degree of independence from other cognitive
  systems. This has the advantage that core abilities are universal and
  are effortlessly acquired with very little experience. One such
  neurological research supports the independence of core systems and
  shows evidence that these domain-specific abilities can be lost while
  other cognitive abilities remain intact.
\item
  core knowledge is made up of a limited number of domain-specific
  systems that comprise evolutionarily important abilities. Currently
  there is evidence supporting domain-specific systems of objects,
  number, agents and their actions, and geometry of the environment.
\end{itemize}

From the above we could conclude that \emph{Piaget's theory of
development} focusses on just \emph{innate learning mechanisms and are
domain general} while \emph{Spelke's theory of core knowledge} focusses
on \emph{particular content knowledge and are domain specific}. nection
between cognitive and biological development. He claimed that knowledge
from social interactions stimulate cognitive growth and development.
Social and cultural factors are influential in the development of
intelligence and passing on of the \emph{history of social experience}
and the use of \emph{cultural tools} are an important for thought
development.

    Post \emph{Piaget} there is no grand unifying theory of cognitive
development. However there have been several trends post piaget in
developmental reasearch:

\begin{itemize}
\tightlist
\item
  connectionism (Bates, Elman, McClelland, Munakata)
\item
  information processing approaches (Case, Klahr, Siegler)
\item
  modularity theories (Leslie)
\item
  core knowledge (Spelke, Carey)
\item
  theory theory (Spelke, Carey, Gelman, Gopnik, Meltzoff,Wellman)
\end{itemize}

We will look at \emph{core knowledge(Spelke, Carey)} as that satisfies
our purpose of question. Core knowledge theory addresses that human
reasoning is guided by a collection of innate domain specific systems of
knowledge. According to \emph{Spelke} children are like \emph{primate
scientists} and argues that children are much more advanced in their
thinking than \emph{Piaget} estimated. These core knowledge systems are
innate mechanisms that do specific tasks. *For example: a songbird's
ability to learn a song characteristic of their species or an ant's
ability to navigate a terrain in search of food, and a human's ability
to acquire language for that matter.

Core knowledge theory operates on two domains:

\begin{itemize}
\tightlist
\item
  \textbf{The domain of objects}
\item
  Experiments with infants show that from an early age infants parse the
  visual array into objects and events. As early as 2 months of age,
  infants expect objects to be permanent.
\end{itemize}

\emph{For Example: If a ball is kept out of sight of an infant while
playing he will look for the ball around him.}

\begin{itemize}
\tightlist
\item
  and that two objects cannot occupy the same space at the same time.
\end{itemize}

\emph{For example: a ball cannot pass through a wall.}

Objects are perceived as bounded, solid, continuous entities.

\begin{itemize}
\item
  \textbf{The domain of number}
\item
  Similar to our knowledge of objects we also posses knowlege of
  knowledge. Experiments with infants show that from as early as 6
  months of age, infants can detect a 1:2 ratio change in large numbers,
  but not a 2:3 ratio changen i.e.~they would be able to tell the
  diffrence between 10 balls and 20 marbles but not between the 20 and
  30 marbles. By 9 months of age, infants can detect a 2:3 ratio change
  and the precision in detection increases to a 7:8 ratio in adults.
\end{itemize}

    The core knowledge domains have mainly three characteristics: - core
knowledge is task specific such that each system functions to solve a
limited set of problems. \emph{For example, the number system can
discriminate quantities based on their ratio, but this is not equivalent
to the ability to count. The natural number system that supports
counting is a complex construction that requires integration from the
object and number systems as well as knowledge about the relation
between number words and their referents.}

\begin{itemize}
\item
  core knowledge domains are summarised in such a way that each system
  operates with a fair degree of independence from other cognitive
  systems. This has the advantage that core abilities are universal and
  are effortlessly acquired with very little experience. One such
  neurological research supports the independence of core systems and
  shows evidence that these domain-specific abilities can be lost while
  other cognitive abilities remain intact.
\item
  core knowledge is made up of a limited number of domain-specific
  systems that comprise evolutionarily important abilities. Currently
  there is evidence supporting domain-specific systems of objects,
  number, agents and their actions, and geometry of the environment.
\end{itemize}

From the above we could conclude that \emph{Piaget's theory of
development} focusses on just \emph{innate learning mechanisms and are
domain general} while \emph{Spelke's theory of core knowledge} focusses
on \emph{particular content knowledge and are domain specific}.

    \hypertarget{section-2}{%
\subsubsection{Section 2}\label{section-2}}

\hypertarget{why-do-children-assume-that-transitive-constructions-are-telic-and-intransitive-constructions-atelic-how-do-you-think-they-grow-out-of-this-bias}{%
\paragraph{Why do children assume that transitive constructions are
telic and intransitive constructions atelic? How do you think they grow
out of this
bias?}\label{why-do-children-assume-that-transitive-constructions-are-telic-and-intransitive-constructions-atelic-how-do-you-think-they-grow-out-of-this-bias}}

The results from the study \emph{Aspectual Bootstrapping in Language
Acquisition: Telicity and Transitivity} conducted by \emph{Laura Wagner}
suggest that children as old as 2 to 3 years use \textbf{transtivity} as
a structural cue to \textbf{telicity semantics} but the study also
suggests that dependence on this cue fades with age.

\textbf{Why this dependence fades away as children grow older?}

Before diving into this we should first understand what determines
\emph{telicity} and how \emph{Laura Wagner} carried out these studies.

    \textbf{The Telic and Atelic distinction}

Telicity is lexical semantic property of predicates and it indicates
whether or not an event has an inherent endpoint or a boundary.

\begin{itemize}
\tightlist
\item
  An event is telic if it has an endpoint.
\item
  An event is atelic if it lacks an endpoint.
\end{itemize}

\begin{longtable}[]{@{}cc@{}}
\toprule
Telic & Atelic\tabularnewline
\midrule
\endhead
The dog jumped into the river. & The dog jumped.\tabularnewline
\bottomrule
\end{longtable}

    \textbf{There are various tests to distinguish between telic and atelic
predicates.}

Let's have a closer look on these tests:

\begin{itemize}
\tightlist
\item
  \textbf{Verb `almost'} \emph{(Dowty, 1979; Vendler, 1967)}
\end{itemize}

When we use the verb \emph{almost} with such predicates, we get two
interpretations for telic predicates and one interpretation for atelic
predicates.

Let's test this against a telic predicate and atelic predicate.flexible
about where their count- ing criterion comes from

\begin{longtable}[]{@{}cc@{}}
\toprule
Telic & Atelic\tabularnewline
\midrule
\endhead
The boy almost ate an apple. & The boy almost ate.\tabularnewline
\bottomrule
\end{longtable}

\emph{In the above example we get two interpretations for the telic
predicate. First, that `The boy started to eat the apple.' and Second,
that `The boy started to eat the apple but did not finish it'. While in
the atelic predicate we notice that we get only one interpretation which
is `The boy did not even begin eating.'}

Clearly, we can see that verb \emph{almost} disambiguates \emph{telic
and atelic} predicates.

    \begin{itemize}
\tightlist
\item
  \textbf{Adverbial `for X time' and `in X time'} \emph{(Dowty, 1979;
  Vendler, 1967)}
\end{itemize}

Telic and atelic naturally combine with different adverbs of time
indicating duration of an event.

\begin{longtable}[]{@{}cc@{}}
\toprule
\begin{minipage}[b]{0.53\columnwidth}\centering
Telic\strut
\end{minipage} & \begin{minipage}[b]{0.41\columnwidth}\centering
Atelic\strut
\end{minipage}\tabularnewline
\midrule
\endhead
\begin{minipage}[t]{0.53\columnwidth}\centering
The boy ate an apple in 5 minutes/\#for 5 minutes.\strut
\end{minipage} & \begin{minipage}[t]{0.41\columnwidth}\centering
The boy ate for 5 minutes/\#in 5 minutes.\strut
\end{minipage}\tabularnewline
\bottomrule
\end{longtable}

\emph{In the example we notice that the telic predicate naturally
combines with the adverbial `in 5 minutes' while, atelic predicate
naturally combine with the adverbial `for 5 minutes'. However, the
non-natural alternatives marked with \# may give a sensible
interpretation.}

    \begin{itemize}
\tightlist
\item
  \textbf{Progressive}
\end{itemize}

\begin{longtable}[]{@{}cc@{}}
\toprule
a. Imperfective-atelic & b. Perfective-atelic\tabularnewline
\midrule
\endhead
Tamyra was singing. & Tamyra sang.\tabularnewline
\bottomrule
\end{longtable}

\emph{In the above examples we notice that the imperfective-atelic
predicate entails perfective-atelic predicate while imperfective-telic
predicate does not entails perfective-telic predicate, instead it
contradicts perfective-telic predicate on adding continuation `when
Simon cut her off', however, adding the continuation to the
imperfective-telic predicate yileds a perfectly natural sounding
construction}

    \begin{longtable}[]{@{}cc@{}}
\toprule
\begin{minipage}[b]{0.48\columnwidth}\centering
c. Imperfective-telic\strut
\end{minipage} & \begin{minipage}[b]{0.46\columnwidth}\centering
d. Perfective-telic\strut
\end{minipage}\tabularnewline
\midrule
\endhead
\begin{minipage}[t]{0.48\columnwidth}\centering
LaToya was singing the song./when Simon cut her off.\strut
\end{minipage} & \begin{minipage}[t]{0.46\columnwidth}\centering
LaToya sang the song./when Simon cut her off.\strut
\end{minipage}\tabularnewline
\bottomrule
\end{longtable}

    \begin{itemize}
\tightlist
\item
  \textbf{Temporal modification} \emph{(Verkuyl,1972,1993)}
\end{itemize}

There are two conditions concerning temporal modification that have to
be fulfilled as suggested by \emph{Verkuyl}. - full temporal PPs should
be conjoined (i.e.~the second `on' cannot be omitted) - the temporal
units denoted by these PPs should be subsequent. \emph{For instance, the
expression `on Monday and on Tuesday' provides good grounds for testing,
while `on Monday and on Wednesday' does not.}

\begin{longtable}[]{@{}cc@{}}
\toprule
\begin{minipage}[b]{0.45\columnwidth}\centering
Atelic\strut
\end{minipage} & \begin{minipage}[b]{0.49\columnwidth}\centering
Telic\strut
\end{minipage}\tabularnewline
\midrule
\endhead
\begin{minipage}[t]{0.45\columnwidth}\centering
Mary drove her car on Monday and on Tuesday.\strut
\end{minipage} & \begin{minipage}[t]{0.49\columnwidth}\centering
Mary ran a mile on Monday and on Tuesday\strut
\end{minipage}\tabularnewline
\bottomrule
\end{longtable}

\emph{In the above examples we notice that atelic predicate gives rise
to a interpretation that `Mary started driving her car on Monday and
continued till Tuesday while telic predicate has only one interpretation
that 'Mary ran 1 mile on Monday and 1 mile on Tuesday' and is
specifically marked by a boundary i.e. `a mile'}

    \begin{itemize}
\tightlist
\item
  \textbf{Countability} \emph{(Bach, 1986)}
\end{itemize}

\emph{Bach} argued that atelic predicates could not be naturally counted
at all but \emph{Wagner} disagrees with him and argues that it is more
difficult to count atelic events than telic ones because atelic
descriptions do not provide criteria for counting---such as an inherent
endpoint but are flexible about where their counting criterion comes
from while telic ones have a specific criterion.

\begin{longtable}[]{@{}cc@{}}
\toprule
Telic & Atelic\tabularnewline
\midrule
\endhead
Ryan fell asleep 3 times & ?? Ryan slept 3 times\tabularnewline
\bottomrule
\end{longtable}

\emph{In the above example, since atelic predicates lack specific
endpoint therefore counting them is more awkward.}

    Despite the complexities of the telicity predicates transitive
structures do play an important role in determining telicity atleast in
the case of infants as young as 2 year olds regardless of whether
transitive structures are the only way to convey telicity or their
ability to do so actually depends on deeper semantics and syntactic
properties. \emph{(cf.Hay et al., 1999; Jackendoff, 1996; Pustejovsky,
1995)}

Laura wagner conducted a study on 2, 3 and 5 years olds children using
an event counting task. Her studies contained a salient goal i.e.~the
right individuation unit for a telic description achieved by two or more
actions/steps separated by a spatiotemporal pause (plausible
individuation units for an atelic description)

\emph{For Example: A boy pushes a ball into the can.(in two distinct
spatiotemporal steps/attempts)}

So, here: Goal= to put the ball into the can (telic predicate)
Spatiotemporal steps= steps/attempts made to put the ball into the
can.(Atelic predicate)

Children were shown movies which were described with either a telic or
an atelic predicate, and the dependent measure was whether children
counted the event's goals or steps.

Wagner tested 4 types of transitivity-telicity:

\begin{longtable}[]{@{}cc@{}}
\toprule
\begin{minipage}[b]{0.37\columnwidth}\centering
Canonical\strut
\end{minipage} & \begin{minipage}[b]{0.57\columnwidth}\centering
All transitive pairs\strut
\end{minipage}\tabularnewline
\midrule
\endhead
\begin{minipage}[t]{0.37\columnwidth}\centering
transitive structure= telic meaning\strut
\end{minipage} & \begin{minipage}[t]{0.57\columnwidth}\centering
transitive structure = telic/atelic meaning\strut
\end{minipage}\tabularnewline
\begin{minipage}[t]{0.37\columnwidth}\centering
intransitive structure= atelic meaning\strut
\end{minipage} & \begin{minipage}[t]{0.57\columnwidth}\centering
\emph{Example1:The bird popped the balloon.}\strut
\end{minipage}\tabularnewline
\begin{minipage}[t]{0.37\columnwidth}\centering
\emph{Example1:The girl painted a flower.}\strut
\end{minipage} & \begin{minipage}[t]{0.57\columnwidth}\centering
\emph{Example2:The bird poked the balloon}\strut
\end{minipage}\tabularnewline
\begin{minipage}[t]{0.37\columnwidth}\centering
\emph{Example2:The girl painted.}\strut
\end{minipage} & \begin{minipage}[t]{0.57\columnwidth}\centering
\strut
\end{minipage}\tabularnewline
\bottomrule
\end{longtable}

\begin{longtable}[]{@{}ll@{}}
\toprule
\begin{minipage}[b]{0.50\columnwidth}\raggedright
All intransitive pairs\strut
\end{minipage} & \begin{minipage}[b]{0.44\columnwidth}\raggedright
Mass count noun pairs\strut
\end{minipage}\tabularnewline
\midrule
\endhead
\begin{minipage}[t]{0.50\columnwidth}\raggedright
instransitive structure = telic/atelic meaning\strut
\end{minipage} & \begin{minipage}[t]{0.44\columnwidth}\raggedright
transitive structure with mass count noun = telic/atelic meaning\strut
\end{minipage}\tabularnewline
\begin{minipage}[t]{0.50\columnwidth}\raggedright
Example1:The door closed.\strut
\end{minipage} & \begin{minipage}[t]{0.44\columnwidth}\raggedright
Example1:The girl drank a glass of juice.\strut
\end{minipage}\tabularnewline
\begin{minipage}[t]{0.50\columnwidth}\raggedright
Example2:The door slid.\strut
\end{minipage} & \begin{minipage}[t]{0.44\columnwidth}\raggedright
Example2:The girl drank juice.\strut
\end{minipage}\tabularnewline
\bottomrule
\end{longtable}

    The results of the study conducted by Wagner showed that children a
syoung as 2 year 10 months old use the telicity of a description to
choose their individuation strategy and that children of this age know
at least some ways of marking telicity in English.

\textbf{Now the question arises as to why do children assume that
transitive constructions are telic and intransitive constructions atelic
and this bias wanes with age?} To answer this question let's not forget
that we are dealing with the children of the age group 2-5 year olds.
However, telicity is one of the topics of semantics which even adults
find it difficult adjust with. Considering the complexities involved in
telicity semantics it is expected as children get older, the reliance on
transitive structures for telicity fades away because children begin to
understand the meanings of the specific elements in the sentence better
as well as how to put those meaning together.

We should also not forget that very young children produce incomplete
predicates so the assessment of knowledge of telicity semantics is based
solely on the available information \emph{i.e.~the verb}.

The studies conducted by \emph{VanHout(2000)} suggests that there is a
long gap between the age at which children notice something about the
and the age at which they know how to syntactically encode it. One study
conducted by \emph{Melissa Kline, Jesse Snedeker \& Laura Schulz(2017)}
on children of 3-4 year old to determine \emph{How do children map
linguistic representations onto the conceptual structures that they
encode?} reveal that children's expectations about transitive verbs are
at least partly driven by their non-linguistic understanding of causal
events. Children expect transitive syntax to refer to scenes where the
agent's action is a plausible cause of the outcome. However, we see that
young infants do differentiate between transitive and atelic predicates
and have rich representation of what constitutes an agent, a cause or an
event but not necessarily that they have adult-like representations.

There could be two factors involved as to why transtivity serves as a
cue to telicity semantics as disscussed by \emph{Wagner}:

\begin{itemize}
\tightlist
\item
  Since direct objects can define event boundaries and serve to measure
  them out.They may not be the only way to accomplish this, and if they
  lack certain properties they will not do it at all, but on the whole
  they are very effective at bounding events.
\item
  The second factor is that the acquisition of meaning is hard
  (cf.~Naigles, 2002), and an imperfect structural cue to meaning is
  better than no cue at all.
\end{itemize}

At younger stage children assume that transitive structures signal telic
meanings, they do not conversely assume that intransitive structures
signal atelic meanings. Children are quite successful at assigning telic
meaning to (surface) intransitive sentences.

    \hypertarget{references}{%
\subsubsection{References}\label{references}}

\begin{itemize}
\tightlist
\item
  Spelke,Elizabeth S. and D. Kinzler,Katherine(2007).Core
  knowledge.\emph{Developmental Science} 10:1 (2007), pp 89--96
\item
  Schulz,Laura(2012).Infant and Early Childhood Cognition.Massachusetts
  Institute of Technology: MIT OpenCourseWare, https://ocw.mit.edu.
\item
  Susan,Hespos (2017).Core knowledge
\item
  Rakison, David(2017).Principles of Child Development.85-221
\item
  Kirk,Sara (2012).Piaget \& Vygotsky: A Comparison.
\item
  Joordens,Steve (2017). Introduction to Psychology.University of
  Toronto, https://www.coursera.org.
\item
  Spelke,Elizabeth S.,Breinlinger,Karen. Macomber,Janet and
  Jocobson,Kristen(1992). Origins of Knowledge, \emph{Psychological
  Review} 1992,Vol.99.No.4,605-632
\item
  Lakusta,Laura \& Susan,Carey(2015) Twelve-Month-Old Infants' Encoding
  of Goal and Source Paths in Agentive and Non-Agentive Motion Events,
  \emph{Language Learning and Development}, 11:2, 152-175, DOI:
  10.1080/15475441.2014.896168
\item
  Wagner, Laura. (2006). Aspectual Bootstrapping in Language
  Acquisition: Telicity and Transitivity. \emph{Language Learning and
  Development}. 2. 51-76. 10.1207/s15473341lld0201\_3.
\item
  Katherine Messenger, Sylvia Yuan \& Cynthia Fisher (2015) Learning
  Verb Syntax via Listening: New Evidence From 22-Month-Olds,
  \emph{Language Learning and Development}, 11:4, 356-368, DOI:
  10.1080/15475441.2014.978331
\item
  Melissa Kline, Jesse Snedeker \& Laura Schulz (2017) Linking Language
  and Events: Spatiotemporal Cues Drive Children's Expectations About
  the Meanings of Novel Transitive Verbs, \emph{Language Learning and
  Development}, 13:1, 1-23, DOI: 10.1080/15475441.2016.1171771
\item
  Lucia Pozzan, Lila R. Gleitman \& John C. Trueswell (2016) Semantic
  Ambiguity and Syntactic Bootstrapping: The Case of Conjoined-Subject
  Intransitive Sentences, \emph{Language Learning and Development},
  12:1, 14-41, DOI: 10.1080/15475441.2015.1018420
\item
  Shaffer,David R. Kipp,Katherine(2007). Theories of Human
  Development.\emph{Developmental Psychology Childhood and Adolescence}.
  pp 41-75,
\end{itemize}

    \begin{Verbatim}[commandchars=\\\{\}]
{\color{incolor}In [{\color{incolor} }]:} 
\end{Verbatim}


    % Add a bibliography block to the postdoc
    
    
    
    \end{document}
